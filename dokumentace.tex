\documentclass[12pt,a4paper,titlepage,final]{article}
\usepackage[czech]{babel}
\usepackage[utf8]{inputenc}

\usepackage[bookmarksopen,colorlinks,plainpages=false,urlcolor=blue,linkcolor=black]{hyperref}
\usepackage{url}
\usepackage{amsmath}
\usepackage{dsfont}
\usepackage{caption}

\usepackage[dvips]{graphicx}
\usepackage[top=3cm, left=3cm, text={15cm, 24cm}, ignorefoot]{geometry}

\newcounter{defcounter} \setcounter{defcounter}{0}
\newenvironment{definition}[1][Definice]{\begin{trivlist}
\item[\hskip \labelsep \refstepcounter{defcounter} {\bfseries #1} {\bfseries \arabic{defcounter}}.]}{\end{trivlist}}

\newenvironment{theorem}[1][Věta]{\begin{trivlist}
\item[\hskip \labelsep {\bfseries #1}]}{\end{trivlist}}

\newcommand{\qed}{\hfill \mbox{\raggedright \rule{.07in}{.1in}}}
\newenvironment{proof}{\vspace{1ex}\noindent{\bf Proof}\hspace{0.5em}}{\hfill\qed\vspace{1ex}}

\begin{document}
\begin{titlepage}
% titulna strnka
\begin{center}

    \vspace{5mm}
    {\LARGE \bf \sc \MakeUppercase{Fakulta informačních technologií} }\\[5mm]%
    {\Large \bf \sc \MakeUppercase{Vysoké učení technické v Brně} }\\[5mm]%
    \vspace{10mm}
\begin{figure}[h]
    \begin{center}
      \includegraphics[width=4cm,keepaspectratio]{img/logo}
    \end{center}
\end{figure}
\vspace{20mm}
{\Huge \bfseries Projekt do předmětu THE}\\[5mm]
{\Large Bayesovské hry}

\vfill
\vfill
\begin{flushleft}
    Tomáš Marek - xmarek11 \hfill \today 
\end{flushleft}
\end{center}

\end{titlepage}

\newpage

\tableofcontents
\pagenumbering{arabic}
\setcounter{page}{1}

\newpage

\section {Úvod} \label{uvod}
Cílem toho to projektu je vytvořit krátký přehled problematiky Bayesovských her, tedy her s~neúplnou informací.
Tato problematika je velmi zajímavou, a to nejen z~hlediska teorie her jako vědy, ale také z~hlediska možné aplikace na reálné problémy skutečného světa, ve kterém je řešení problémů při nedostatku, nebo nekompletnosti informací velmi častou praxí.
Uveďme několik příkladů:
\begin{itemize}
 \item V~aukcích neznají zpravidla jednotliví dražitelé hodnocení všech dražených položek všemi ostatními dražiteli
 \item Při přijímání nových zaměstnanců zaměstnavatel nezná schopnosti uchazečů o~zaměstnání
 \item V~Cournotově modelu oligopolu nemusejí jednotliví oligopolisté znát nákladové struktury ostatních oligopolistů(ukázka tohoto problému bude řešena jako příklad)
 \item Postoje jednotlivých hráčů k~riziku nemusí být všeobecně známé
\end{itemize}
Ukážeme si, jakým způsobem fungují Bayesovské hry, co jsou a jak najít ekvilibria ve hře.

Teorii Bayesovských her poprvé představil Maďarsko--americký ekonom John Harsanyi ve své publikaci \textit{Games with Incomplete Information Played by "Bayesian"}\cite{Harsanyi2004}.
Za tuto teorii v~roce 1994 získal Nobelovu cenu za "průkopnickou analýzu ekvilibrií v~teorii nekooperativních her"\footnote{\url{http://www.nobelprize.org/nobel_prizes/economics/laureates/1994/harsanyi.html}}.

Úvodem této práce bychom měli ještě zmínit, že notace, nadále používaná v~této práci, vychází z~notace používané panem PhD. Martinem Hrubým Ing. při výuce předmětu \textit{Teorie her} na Fakultě informatiky Vysokého učení technického v~Brně.

\section{Bayesovské hry}

Bayesovské hry jsou nekooperativní hry v~normální formě, ve kterých hráči, jak již bylo zmíněno v~úvodu, nemají kompletní informace o~hře -- mají určitou nejistotu například o~užitkových funkcích, o~možných strategiích ostatních hráčů atd.
Tato vlastnost Bayesovských her je aspektem, jenž se nedá v~teorii her pomocí běžně používaných přístupů příliš dobře simulovat a řešit.

Dle Harsanyiho můžeme rozdělit hry na \textit{hry s~kompletní informací}, občas nazývány C--games a \textit{hry s~nekompletní informací}, také nazývány I--games\footnote{Hry s~kompletní informací a hry s~nekompletní informací (tedy C--games a I--games) by se neměly zaměňovat se hrami s~dokonalou a nedokonalou informací (games with incomplete information a games with imperfect information). Rozdělení her na C--games a I--games se vztahuje vzhledem ke znalostem jednotlivých hráčů o~hře, na druhou stranu rozdělení her na hry s~dokonalou a nedokonalou informací se vztahují ke množství informací, které hráč má o~ostatních hráčích (histiorii jejich tahů atd.). Na rozdíl od her s~nekompletní informací jsou hry s~nedokonalou informací velmi dobře dokumentovány v~literatuře.}.
Hry s~nekompletní informací jsou význačné tím, že na rozdíl od her s~kompletní informací, chybí hráčům kompletní informace o~hrané hře.
Hráči například nemají informace o~užitkových funkcích jiných hráčů, o~strategiích ostatních hráčů, ale může se také stát, že nemají úplné informace sami o~sobě, kupříkladu neznají svou užitkovou funkci.
Obor teorie her se téměř výhradně zabývá jenom hrami s~kompletními informacemi a to i přesto, že ve skutečné ekonomice, politice, ve vojenství a v~dalších životních situacích je neznalost faktů a i přes to nutnost řešení problémů velmi častá.

Harsanyi ve své práci dále tvrdí, že důvodem, proč se historicky odborníci oboru teorie her příliš nezabývali hrami s~nekompletní informací byla zdánlivá nekonečnost modelů her, způsobena vzájemnými očekáváními hráčů.
Tuto vlastnost si představme na příkladě hry dvou hráčů ($H_1$ a $H_2$), ve které každý hráč zná svoji vlastní užitkovou funkci, ale nezná užitkovou funkci protihráče.
V~takovéto hře by hráč $H_1$ volil strategii v~závislosti na tom, jakou by očekával užitkovou funkci $u_2$ hráče $H_2$.
Tomuto očekávání se také říká očekávání první třídy (first-order expectation).
Volba jeho strategie hráče $H_1$ ale nebude záviset pouze na jeho vlastním očekávání prvního řádu, ale také na očekávání prvního řádu hráče $H_2$ snažícího se odhadnout užitkovou funkci hráče $H_1$.
Tomuto očekávání zabývajícím se očekáváními prvního řádu se říká očekávání druhého řádu (second-order expectation).
Takto to pokrčuje dál, tedy strategie volená hráčem $H_1$ bude také záviset na jeho očekávání, jaké bude očekávání druhého řádu hráče $H_2$.
Tomuto se tedy říká očekávání třetího řádu.
Tento postup by takto mohl pokračovat do nekonečna.
Stejným způsobem potom bude hráč $H_2$ určovat svoji strategii, tedy v~závislosti na svých očekáváních prvního řádu, druhého řádu, třetího řádu atd. až do nekonečna.
Tomuto modelu her s~nekompletní informací se říká \textit{model sekvenčních očekávání} (sequential exectations model).

V~Bayesovských hrách má každý hráč vlastní subjektivní pravděpodobnostní rozdělení pro charakteristické informace ostatních hráčů, které jsou mu neznámé.
Harsanyi ve své publikaci navrhl pro řešení Bayesovských her používat přírody (v~teorii her je příroda používá pro neinteligentního hráče jednajícího zcela náhodně) pro přiřazování typů hráčům na základě základního pravděpodobnostního rozložení (viz dále) a převést tak hru s~nekompletními informacemi na hru se sice nedokonalými, ale kompletními informacemi.
V~Bayesovských hrách existuje takzvané základní pravděpodobnostní rozložení (basic probablity distribution)\cite{Harsanyi2004} určující charakteristiky hráčů tedy užitkovou funkci.
Kompletní soukromé informace každého hráče definují jeho typ.
Každý hráč má o~sobě kompletní informace (zná svoji užitkovou funkci, možnosti, atd.), a tedy zná svůj vlastní typ.
Typy ostatních hráčů jsou však hráči neznámé.
Jelikož však každý hráč zná také základní pravděpodobnostní rozložení, může si tedy vytvořit vlastní předpoklady o~typech ostatních hráčů a podle těchto předpokladů následně hrát. 

\subsection{Bayesovská hra}
\begin{definition}
 \textbf{Bayesovské hra} -- Bayesovská hra $H$ je určena jako \\
 $(Q, \{S_i\}_{i \in Q}, \{T_i\}_{i \in Q}, \{p_i\}_{i \in Q}, u)$, kde
\begin{itemize}
 \item $Q$ je množina hráčů $Q = \{1,2,\dots , N\}$
 \item $S_i, i \in Q$ jsou množiny ryzích strategií hráče $i \in Q$. Prvek prostoru $S_k$ budeme značit $s_k$
 \item $T_i, i \in Q$ jsou množiny všech typů hráčů ve hře. Typ $t_i \in T_i$ odpovídá určité výplatní funkci, kterou může mít hráč $i$. Hráč $i$ zná svůj typ, ale nezná typy ostatních hráčů.
 \item $P_i, i \in Q$ je množina názoru hráčů. $p_i \in P_i$ představuje názor hráče $i$ na typy ostatních hráčů.
 \item užitková funkce $u$, která každé uspořádané dvojici $(a_i, t_i) \in S_i \times T_i $ přiřadí hodnotu z~$\mathds{R}$ pro $i \in Q$
\end{itemize}
\end{definition}
Takto definovaná Bayesovská hra je tedy rozšířením strategické nekooperativní hry N hráčů o~množiny typů hráčů $T_i$ a množiny odhadů $P_i$ a s~upravenou užitkovou funkcí.
Když volí hráč $i$  svou strategii, volí ji podle svých předpokladů o~typech protihráčů.
Strategiemi jsou tedy myšleny dvojice $(s_i, t_i) \in T_i)$.
Jestliže jsou množiny $Q$, $S$ a $T$ konečné, pak tuto Bayesovskou hru nazýváme \textit{konečnou}.

Ryzí strategií $s_i$ hráče $i$ je v~Bayesovské hře funkce mapující hráčův typ do množiny strategií.
$$ s_i : T_i \rightarrow S_i $$
Pravděpodobnost rozdělení při strategii $p(t_i)$ a pravděpodobnost hraní strategie $s_i$ hráčem $i$ se bude označovat jako $p_i(s_i|t_i)$, neboli pravděpodobnost přiřazená funkcí $p$ strategii $s_i$ za předpokladu typu $t_i$.

\subsection{Bayesovské Nashovo ekvilibrium}
Ačkoliv jsou Bayesovské hry poměrně hodně odlišné od ostatních typů her v~teorii her, princip Nashova ekvilibria zůstává neměnný. 
Abychom však mohli nadefinovat Bayesovské Nashovo ekvilibrium, musíme nejprve nadefinovat best response hráče v~Bayesovských hrách.
K~tomu musíme ale nejprve vědět, jak vypočítat očekávaný užitek $E_t[u_i(s(t),t)]$ ze hry daného hráče $i$.
Ten lze spočítat podle následujícího vzorce:
$$ E_t[u_i(s(t),t)] = \sum_{t_i \in T_i}p(t_i){E_t}_{-i} [u_i(st),t|t_i)] $$
Díky tomu, že jsme již schopni vypočítat očekávaný užitek, můžeme nyní nadefinovat best response, neboli nejlepší odpověď hráče $i$.
\begin{definition}
 \textbf{Best response} -- Uvažujme Bayesovskou hru, hráče $i \in Q$, konkrétní sub-profil strategií $s_{-i} \in S_{-i}$ a konkrétní sub-profil typů ${t_{-i} \in T_{-i}}$. Nejlepší odpověď hráče $i$ v~situaci $s_{-i}(t_{-i})$ je:

$$BR_i(s_{-i}(t_{-i})) = arg \max_{s_i(t_i) \in S_i} E_t[u_i(s_i(t_i), s_{-i}(t_{-i}), t)]$$
$$BR_i(s_{-i}(t_{-i})) = \sum_{t_i \in T_i} p(t_i)(arg \max_{s_i(t_i) \in S_i} {E_t}_{-i}[u_i(s_i(t_i), s_{-i}(t_{-i}), t)|t_i])  $$
\end{definition}

Nyní když máme definovánu funkci best response můžeme přikročit k~definici Bayesovského Nashova ekvilibria.

\begin{definition}
  \textbf{Bayesovské Nashovo ekvilibrium} -- je smíšený profil ${\sigma}^* = ({{\sigma}^*}_i)_{i \in N}$ takový, že pro každého hráče $i \in Q$ a pro každé $t_i \in T_i$ plat, že
$${\sigma}^*(t_i) \in BR_i$$
  
\end{definition}

Jelikož nyní již dokážeme spočítat užitek hráče a jsme schopni také nalézt Bayesovské Nashovo ekvilibrium, můžeme nyní přejít do další kapitoly ve které budou řešeny základní příklady Bayesovských her.

\section{Příklady}
\subsection{Cournotův model duopolu s~nekompletní informací}
Mějme dvě firmy -- firmu 1 a firmu 2.
U~těchto firem předpokládejme výrobu homogenních výrobků.
Dále předpokládejme, že firmy produkují zboží s~náklady $c_1, c_2 \geq 0.$
Užitkové funkce firmy $i$, $i \in {1, 2}$ jsou tedy:
$$u_i(q_1, q_j) = P(q_1 + q_2)q_i - c_i q_i = [ 2 - q_1 - q_2 - c_i] q_i$$
Pokud by byly výrobní náklady známy, řešil by se problém jako klasický Cournotův duopol.
Nicméně předpokládejme, že:
\begin{itemize}
 \item Firma 1 má výrobní náklady $c_1 = 1$. Tato vlastnost je common knowlidge.
 \item Firma 2 má výrobní náklady $c_2 = \frac{1}{2}$ s~pravděpodobností $\frac{1}{2}$ a $c_2 = \frac{3}{2}$ s~pravděpodobností $\frac{1}{2}$
\end{itemize}

Předpokládejme nyní, že $q_1$ je množství výrobků zvoleno firmou $1$.
Optimální množství vyrobených výrobků firmy $2$ je tedy:

$$ {q_2}^L = arg max [2 - q_1 - q_2 - \frac{1}{2}]q_2 = arg max[\frac{3}{2} - q_1 - q_2]q_2 $$

pro $c_2 = \frac{1}{2}$ a:

$$ {q_2}^H = arg max [2 - q_1 - q_2 - \frac{3}{2}]q_2 = arg max[\frac{1}{2} - q_1 - q_2]q_2 $$

pro $c_2 = \frac{3}{2}$.

Nyní hledáme best response firmy 2 jak pro ${q_2}^L$, tak pro ${q_2}^H$.
Pro výpočet této hodnoty hledáne maximum těchto funkcí (postup v~\cite{bgames-hrubym}, Řešení oligopolů).
Výsledné best response tedy jsou:

$$
{q_2}^L = \frac{\frac{3}{2} - q_1}{2}
$$
$$
{q_2}^H = \frac{\frac{1}{2} - q_1}{2}
$$

Firma 1 nyní maximalizuje svůj očekávaný profit na základě vypočtených ${q_2}^L$ a ${q_2}^H$:
\begin{align*}
q_1({q_2}^L, {q_2}^H) &= arg max_{q_1} \frac{1}{2}[2 - q_1 - {q_2}^L - 1]q_1 + \frac{1}{2}[2 - q_1 - {q_2}^H - 1] q_1 \\
&= arg max_{q_1} [1 - q_1 - (\frac{1}{2}{q_2}^L + \frac{1}{2}{q_2}^H)]q_1 \Rightarrow
\end{align*}
Pro tuto funkci budeme taktéž hledat její maximum tzn. získáme:
$$
q_1({q_2}^L, {q_2}^H) = \frac{1 - (\frac{1}{2} {q_2}^L + \frac{1}{2} {q_2}^H)}{2}
$$
Nyní přistupme k~výpočtu Nashova ekvilibria.

\begin{align*}
{q_1}^* &= q_1({q_2}^L, {q_2}^H) = \frac{2 - {q_2}^L - {q_2}^H)}{4} = \frac{2 - \frac{\frac{3}{2} - q_1}{2} - \frac{\frac{1}{2} - q_1}{2}}{4} \\
&= \frac{2 - \frac{2 - 2q_1}{2}}{4} = \frac{2 - 1 + q_1}{4} \Leftrightarrow q_1\frac{3}{4} = \frac{1}{4} \Leftrightarrow {q_1}^* = \frac{1}{3}
\end{align*}

a pro $q_2$:

\begin{align*}
{q_2}^{*L} &= {q_2}^L({q_1}^*) = \frac{\frac{3}{2} - {q_1}^*}{2} = \frac{\frac{3}{2} - \frac{1}{3}}{2} = \frac{7}{12} \\
{q_2}^{*H} &= {q_2}^H({q_1}^*) = \frac{\frac{1}{2} - {q_1}^*}{2} = \frac{\frac{1}{2} - \frac{1}{3}}{2} = \frac{1}{12} \\
 &\Rightarrow \frac{1}{2}{q_2}^{*L} + \frac{1}{2}{q_2}^{*H} = \frac{7}{24} + \frac{1}{24} = \frac{1}{8} = \frac{1}{3}
\end{align*}

tzn. hledané ekvilibrium je $(\frac{1}{3}, \frac{1}{3})$

\subsection{Battle of sexes s nekompletní informací}
Předpokládejme dva hráče $1$ a $2$.
Hráč $2$ má o hře kompletní informaci a může hrát dva typy hry -- \emph{l} a \emph{h}.
Pokud bude hráč $2$ hrát typ \emph{l}, bude to znamenat, že má hráče $1$ rád a tudíž s ním i rád bude.
Naopak pokud bude hrát typ \emph{h}, hráče $1$ mít rád nebude a bude preferovat nebýt s ním.
Pravděpodobnost obou typů hry hráče $2$ je $\frac{1}{2}$.
Hráč $1$ naopak nemá kompletní informaci o hře -- neví, zda ho má, nebo nemá hráč $2$ rád (tzn. neví, zda bude hrát variantu \emph{l}, nebo \emph{h}).

Následující tabulky popisují hodnoty užitků obou hráčů pro jednotlivé typy hry:

\begin{table}[h]
	\parbox{.45\linewidth}{
	\centering
	\begin{tabular}{|l|c|c|}
  		\hline                        
  			 	   & balet & fotbal \\
  		\hline                       
  			balet  & (2,1) & (0,0)  \\
  		\hline                       
  			fotbal & (0,0) & (1,2)  \\
  		\hline  
	\end{tabular}
	\caption{U2žitky pokud bude hráč $2$ hrát typ \emph{l}}
	}
\hfill
	\parbox{.45\linewidth}{
	\centering
	\begin{tabular}{|l|c|c|}
  		\hline                        
  	 			   & balet & fotbal \\
  		\hline                       
  			balet  & (2,0) & (0,2)  \\
  		\hline                       
  			fotbal & (0,1) & (1,0)  \\
  		\hline  
	\end{tabular}
	\caption{Užitky pokud bude hráč $2$ hrát typ \emph{h}}
}
\end{table}

Tuto hru tedy můžeme reprezentovat jako Bayesovskou hru:
\begin{itemize}
	\item $Q = \{1,2\}$
	\item $S = \{B,F\}$
	\item $T_{1} = \{x\}$, $T_{2} = \{l,h\}$
	\item $p_{1}(l|x) = p_{1}(h|x) = \frac{1}{2}$, $p_{2}(x|l) = p_{2}(x|h) = 1$\\
	\item $u_{1}, u_{2}$ viz tabulky výše
\end{itemize}

Jelikož hráč $1$ má pouze jeden typ (jeho typ je common knowledge), hráč $2$ bude mít kompletní informaci o hře a bude hrát hru jako by byla klasická nekooperativní hra s kompletní informací.
Typem hráče $1$ se tedy nadále nebudeme zabývat.

Abychom nalezli Bayesovské ekvilibrium, musíme analyzovat rozhodovací problém pro všechny hráče všech typů a vytvorit asociovanou hru v normální formě s kompletní informací ekvivalentní s původní Bayesovskou hrou.

Vezměme nyní kupříkladu v úvahu strategický profil $(B, (B,F))$.
Tento profil říká, že hráč $1$ hraje strategii $B$, zatímco hráč $2$ hraje také profil $B$, nebo pokud je typu \emph{h}, tak hraje strategii $F$.
Pro tento profil budeme nyní chtít vypočítat užitky obou hráčů, pomocí kterých následně sestavíme onu asociovanou hru z níž budeme schopni určit ekvilibria hry.

Užitek hráče $1$ v profilu $(B, (B,F))$ vypočteme jako:
$$ u_{1}(B,(B,F)) = p_{1}(l|x) * u_1(B,B) + p_{1}(h|x) * u_1(B,F) $$
Tudíž:
$$u_{1}(B,(B,F)) = \frac{1}{2} \cdot 2 + \frac{1}{2} * 0 = 1$$
užitek hráče $2$ potom obdobně:
$$u_{2}(B,(B,F)) = p_{1}(l|x) * u_2(B,B) + p_{1}(l|x) * u_2(B,F) $$
$$u_{2}(B,(B,F)) = \frac{1}{2} \cdot 1 + \frac{1}{2} * 2 = \frac{3}{2}$$

Budeme používat obdobný postup popsaný výše pro všechny zbylé profily, dokud nedostaneme následující tabulku užitků:

\begin{table}[h]
	\centering
	\begin{tabular}{|l|c|c|c|c|}
  		\hline                        
  			 	   & (B,B)               & (B,F)            & (F,B)                         & (F,F) \\
  		\hline                       
			B      &   \underline{2},$\frac{1}{2}$   &  \underline{1},\underline{$\frac{3}{2}$} &  \underline{1},0                          &  0,1 \\
  		\hline                       
  			F      &   0,$\frac{1}{2}$   &  $\frac{1}{2}$,0 &  $\frac{1}{2}$,\underline{$\frac{3}{2}$}  &  \underline{1},1 \\
  		\hline  
	\end{tabular}
	\caption{Užitky hráčů pro hru převedenou na hru s kompletní informací}
\end{table}

Tato tabulka je tedy tabulka užitků hráčů 1 a 2 pro asociovanou nekooperativní hru v normální formě s kompletní informací vycházející ze zadané bayesovské hry.
V této hře budeme standardním způsobem (stejným jako na přednáškách předmětu THE) hledat ekvilibria a získáme tak výsledek, který říká, že jediným ekvilibriem hry je profil \emph{(B, (B, F))}.

\newpage

\section{Závěr} \label{zaver}
V~tomto projektu jsem se zabýval problematikou Bayesovských her, tedy her s~neúplnou informací.
Prostudoval jsem teorii Bayesovských her z~literatury (viz níže) a vytvořil jsem kompilaci toho, co jsem se naučil.
Definoval jsem zde Bayesovské hry, best response hráče $i$ v~situaci $s_{-i}(t_{-i})$  v~těchto hrách a Bayesovské Nashovo Ekvilibrium definující rovnovážný stav ve hrách s nekompletní informací.
Na závěr jsem prezentoval dva řešené příklady využití Bayesovských her -- Cournotův model duopolu s neúplnou informací ve hře a hru souboj pohlaví (Battle of Sexes) s neúplnou informací ve hře.

Toto téma jsem si pro svou práci zvolil především proto, že jsou aplikovatelné na reálné životní situace ve kterých nemáme kompletní informaci o "hře".

\nocite{bgames}
\nocite{bgames-hyksova}
\nocite{bgames-ratliff}
\nocite{bgames-chen}
\nocite{bgames-ozdaglar}
\nocite{bgames-ordonez}

\newpage
\bibliographystyle{czechiso}
\bibliography{dokumentace}

\end{document}
